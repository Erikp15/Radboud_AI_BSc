\documentclass{article}
\usepackage{bm}
\usepackage{amsmath,amsfonts,amssymb,amsthm}
\usepackage{newpxtext} 
\usepackage{relsize}
\usepackage{comment}
\usepackage{booktabs}
\usepackage{stackengine} 
\usepackage{adjustbox}
\usepackage{mathtools}
\usepackage{graphicx} % Required for inserting images

\title{Calculus Assignment 2}
\author{Erik Paskalev}
\date{September 2024}

\begin{document}

\maketitle

\section{\normalfont Show that the power }

\begin{equation}
(t + \Delta t)^4 = (t + \Delta t)(t + \Delta t)(t + \Delta t)(t + \Delta t)
\end{equation}

\section*{\normalfont can be written as}

\begin{equation}
t^4 = 4t^3\Delta t + (...)(\Delta t)^2
\end{equation}

\section*{Answer:}

\begin{equation}
\begin{split}
(t+\Delta t)^4 & = t^4 + 4t^3\Delta t + 6t^2{\Delta t}^2 + 4t{\Delta t}^3 + {\Delta t}^4 = \\ 
& =  t^4 + 4t^3\Delta t + (6t^2 + 4t\Delta t + {\Delta t}^2){\Delta t}^2
\end{split}
\end{equation}

\section{\normalfont Use the previous result to show, without using the power rule, that}

\begin{equation}
(t^4)' = 4t^3
\end{equation}

\section*{Answer:}

\begin{equation}
\begin{split}
\lim_{\Delta t \to 0} \frac{x(t+\Delta t)-x(t)}{\Delta t} & = \lim_{\Delta t \to 0} \frac{(t+\Delta t)^4 - t^4}{\Delta t} = \\ 
& = \lim_{\Delta t \to 0} \frac{t^4 + 4t^3\Delta t + 6t^2{\Delta t}^2 + 4t{\Delta t}^3 + {\Delta t}^4 - t^4}{\Delta t} = \\ 
& = \lim_{\Delta t \to 0} 4t^3 + 6t^2\Delta t + 4t{\Delta t}^2 + {\Delta t}^3 = 4t^3 + 0 + 0 = 4t^3
\end{split}
\end{equation}

\subsection*{The reason the ${\Delta t}^2$ and higher order coefficients don't matter is because $\Delta t = 0$ so once we solve the indeterminate form and replace $\Delta t$ with the limit all coefficients behind $\Delta t$ are multiplied by 0.}

\section{\normalfont Compute the derivative of the following functions using the power rule, the sum rule and the scaling rule.}

\begin{equation}
\begin{split}
t^5 & = 5t^4 \\
5t^4 & = 20t^3 \\
2t + t^5 & = 2 + 5t^4 \\
4t^2 + 2t^7 & = 8t + 14t^6 \\
2 + 3t + 4t^2 + 5t^3 & = 3 + 8t + 15t^2 
\end{split}
\end{equation}

\section{\normalfont Prove the scaling rule:}

\begin{equation}
(cx(t))' = c(x(t))'    
\end{equation}

\section*{\normalfont using the sum rule.}

\section*{Proof:}

\begin{equation}
(cx(t))' = (\underbrace{x(t) + x(t) + ... + x(t)}_{\text{c times}})' = \underbrace{x(t)' + x(t)' + ... + x(t)'}_{\text{c times}} = c(x(t))'
\end{equation}

\end{document}
