\documentclass{article}
\usepackage{bm}
\usepackage{amsmath,amsfonts,amssymb,amsthm}
\usepackage{newpxtext} 
\usepackage{relsize}
\usepackage{comment}
\usepackage{booktabs}
\usepackage{stackengine} 
\usepackage{adjustbox}
\usepackage{mathtools}
\usepackage{graphicx} % Required for inserting images

\title{Calculus Assignment 4}
\author{Erik Paskalev}
\date{September 2024}

\begin{document}

\maketitle

\section{\normalfont Compute the following local linear approximations:}

\begin{equation}
\begin{split}
x(t) & = t^2 \text{ for } t_0 = 1 \\
x'(t) & = 2t \\
\widehat{x}(t) & = x'(t_0)(t-t_0) + x(t_0) = 2t_0t - 2{t_0}^2 + {t_0}^2 = 2t - 1 \\ 
\\
x(t) & = t^2e^t \text{ for } t_0 = 1 \\
x'(t) & = (2t+t^2)e^t \\
\widehat{x}(t) & = x'(t_0)(t-t_0) + x(t_0) = (2t_0 + {t_0}^2)e^{t_0} t - \\ 
& - (2t_0 + {t_0}^2)e^{t_0} t_0 + {t_0}^2e^{t_0} = 3et - 2e \\
\\
x(t) & = 2te^t \text{ for } t_0 = 0 \\
x'(t) & = (t+1)2e^t \\
\widehat{x}(t) & = x'(t_0)(t-t_0) + x(t_0) = (t_0+ 1)2e^{t_0} t +{t_0}^2e^{t_0} = 2t\\
\end{split}
\end{equation}


\section{\normalfont Compute the second derivatives of the following functions:}

\begin{equation}
\begin{split}
x(t) & = 5t \\
x''(t) & = 0 \\ \\
x(t) & = t^2 + 2t \\
x''(t) & = 2 \\ \\
x(t) & = te^t + 2t^2 \\
x''(t) & = (e^t + te^t + 4t)' = 2e^t + te^t + 4 = (t+2)e^t + 4 \\ \\
x(t) & = t^3 3^t \\
x''(t) & = (3t^2 3^t + ln(3)t^3 3^t)' = 6t 3^t + 3ln(3)t^2 3^t + 3ln(3)t^2 3^t + ln(27)t^3 3^t = \\ 
& = (6 + 6ln(3)t + ln(27)t^2)t 3^t
\end{split}
\end{equation}

\section{\normalfont Use the derivative and the second derivative to compute the local quadratic approximation of the following function:}

\begin{equation}
x(t) = t^3e^t
\end{equation}

\section*{\normalfont for $t_0 = -1$.  Plot the function, its local linear approximation and its local quadratic approximation use Geogebra.}

\section*{Answer:}

\begin{equation}
\begin{split}
x'(t) & = 3t^2e^t + t^3e^t\\
x''(t) & = (6 + 6t + t^2)te^t\\
\widehat{x}(t) &= \frac{1}{2}x''(t_0)(t-t_0)^2 + x'(t_0)(t-t_0) + x(t_0) = \\ 
& = \frac{1}{2}(6 + 6t_0 + {t_0}^2)t_0e^{t_0}(t^2 - 2tt_0 + {t_0}^2) + (3{t_0}^2e^{t_0} + {t_0}^3e^{t_0})(t-t_0) + {t_0}^3e^{t_0} = \\ 
& = -\frac{t^2 + 2t + 1}{2e} + \frac{3t + 3}{e} - \frac{t + 1}{e} - \frac{1}{e} = \\ 
& = -\frac{t^2 + 2t + 1 - 6t - 6 + 2t + 2 + 2}{2e} = \\ 
& = -\frac{t^2 - 2t - 1}{2e} = -\frac{1}{2e}t^2 + \frac{1}{e}t + \frac{1}{2e}
\end{split}
\end{equation}

\section{\normalfont Determine the range of input values where the following function is decreasing.}

\begin{equation}
x(t) = t^3 - t
\end{equation}

\section*{Answer:}

\begin{equation}
\begin{split}
x'(t) = 3t^2 - 1 & < 0 \\
t^2 - \frac{1}{3} & < 0 \\
(t - \frac{1}{\sqrt{3}})(t + \frac{1}{\sqrt{3}}) & < 0 \\
t \in (-\frac{1}{\sqrt{3}};\frac{1}{\sqrt{3}}) &
\end{split}
\end{equation}

\section{\normalfont Find all the critical points of the following function:}

\section*{Answer:}

\begin{equation}
\begin{split}
& x(t) = t - t^3 \\
& x'(t) =  1 - 3t^2 = 0 \\
& \frac{1}{\sqrt{3}} - t^2 = 0 \\
& (t - \frac{1}{\sqrt{3}})(t + \frac{1}{\sqrt{3}}) = 0 \\
& \text{Critical points: } A = (-\frac{1}{\sqrt{3}},-\frac{2\sqrt{3}}{9}), B = (\frac{1}{\sqrt{3}},\frac{2\sqrt{3}}{9}) \\
\end{split}
\end{equation}

\begin{equation}
\begin{split}
& x(t) = t^2 - t^4\\
& x'(t) = 2t - 4t^3 = 0 \\
& -(2t^2 - 1)2t = 0 \\
& (t - \frac{1}{\sqrt{2}})(t + \frac{1}{\sqrt{2}})2t = 0 \\
& \text{Critical points: } A = (-\frac{1}{\sqrt{2}},\frac{1}{4}), B = (0, 0), C = (\frac{1}{\sqrt{2}},\frac{1}{4}) \\
\end{split}
\end{equation}

\begin{equation}
\begin{split}
& x(t) = t^2e^t \\
& x'(t) = (t + 2)te^t = 0 \\
& \text{Critical points: } A = (-2,\frac{4}{e^2}), B = (0, 0)\\
\end{split}
\end{equation}

\section{\normalfont Use the second derivative to check if the critical points you found in the previous exercise are points of minimum or points of maximum.}

\section*{Answer:}

\begin{equation}
\begin{split}
& x(t) = t - t^3\\
& x'(t) = 1 - 3t^2 \\
& x''(t) = -6t \\
& x''(-\frac{1}{\sqrt{3}}) = 2\sqrt{3} > 0 \text{ meaning point A is local minimum} \\ 
& x''(\frac{1}{\sqrt{3}}) = -2\sqrt{3} < 0 \text{ meaning point C is local maximum} \\
\end{split}
\end{equation}

\begin{equation}
\begin{split}
& x(t) = t^2 - t^4\\
& x'(t) = 2t - 4t^3 \\
& x''(t) = 2 - 12t^2 \\
& x''(-\frac{1}{\sqrt{2}}) = -4 < 0 \text{ meaning point A is local maximum}\\
& x''(0) = 2 > 0 \text{ meaning point B is local minimum} \\
& x''(\frac{1}{\sqrt{2}}) = -4 < 0 \text{ meaning point C is local maximum} \\
\end{split}
\end{equation}

\begin{equation}
\begin{split}
& x(t) = t^2e^t\\
& x'(t) = (t + 2)te^t \\
& x''(t) = (t + 1)2e^t + (t + 2)te^t = (t^2 + 4t + 2)e^t\\
& x''(-2) = -\frac{2}{e^2} < 0 \text{ meaning point A is local maximum}\\
& x''(0) = 2 > 0 \text{ meaning point B is local minimum} \\
\end{split}
\end{equation}

\end{document}
